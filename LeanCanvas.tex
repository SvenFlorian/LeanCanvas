\documentclass{article}

% Adapted from Alejandro Ochoa's Grant Model Canvas:
% https://github.com/OchoaLab/grantModelCanvas

% His notes regarding inspiration for the code:
%       adapted "business" version from latex code:
%       https://rememberthecmd.blogspot.com/2015/02/draw-business-model-generation-canvas.html
%       and the "grant" version (not latex):
%       https://gregglab.neuro.utah.edu/2018/10/30/the-grant-model-canvas-for-developing-great-grants/


%%%%%%%%%%%%%%%%%%%%%%%%
%%%% DOCUMENT START %%%%
%%%%%%%%%%%%%%%%%%%%%%%%

\usepackage[landscape,margin=0in]{geometry}
%\usepackage[english]{babel}
\usepackage[utf8]{inputenc}
\pagenumbering{gobble} % supress page numbers
\usepackage{hyperref}
\usepackage{datetime2} % YYYY-MM-DD format!
% Sans-serif fonts
\renewcommand{\rmdefault}{phv}
\renewcommand{\sfdefault}{phv} 

\usepackage{tikz}

\title{\vspace{-1em}Lean Canvas}
\author{Florian Minges --- \href{mailto:florianminges@gmail.com}{\nolinkurl{florianminges@gmail.com} } }
\date{
  \textbf{\color{black!50}Revision number:} 001
  ~\textbf{\color{black!50}---}
  \textbf{\color{black!50}Date:} \today
}


\begin{document}
\maketitle

\vspace{-1em}
\centering
\def\layersep{9.7em}
\def\layerwidth{75em}

\makebox[\textwidth][c]{
  \begin{tikzpicture}[
      % Define block parameters (mostly shape)
      bloc/.style={
        rectangle, rounded corners,
        draw=black!30, very thick, inner sep=0,
      },
      invisible/.style={
        rectangle, draw=none,
        inner sep=0,
      },
      bloc1/.style={
        bloc,
        text width = \layerwidth/5*0.95,
        minimum width = \layerwidth/5,
        minimum height= 4*\layersep
      },
      bloc2/.style={
        bloc,
        text width = \layerwidth/5*0.95,
        minimum width=\layerwidth/5,
        minimum height=2*\layersep
      },
      bloc3/.style={
        bloc,
        text width=\layerwidth/2*0.95,
        minimum width=\layerwidth/2,
        minimum height=\layersep
      },
      invisible_bloc1/.style={
        invisible,
        text width=\layerwidth/5*0.95,
        minimum width=\layerwidth/5,
        minimum height=\layersep
      },
      invisible_bloc2/.style={
        invisible,
        text width=\layerwidth/5*0.95,
        minimum width=\layerwidth/5,
        minimum height=3*\layersep
      },
      title/.style={
        anchor=north west,
        color=black!50,
        font=\bfseries
      },
      subtitle/.style={
        anchor=north west,
        color=black!50,
        font=\bfseries
      },
    ]
    
    %%%%%%%%%%%%%%%%%%%%%%%%%
    %%%% DRAW THE CANVAS %%%%
    %%%%%%%%%%%%%%%%%%%%%%%%%

    % first the block, then the title
    
    % 1. PROBLEM
    \node[bloc1] (b1) at (0*\layerwidth/10,4*\layersep) {};
    \node[title] at (b1.north west) {\underline{1. Problem}};
    
    \node[invisible_bloc2] (b11) at (0*\layerwidth/10,4.45*\layersep) {
      List your top 1-3 problems.
    };
    
    \node[invisible_bloc1] (b12) at (0*\layerwidth/10,2.5*\layersep) {
      List how these problems are solved today.
    };
    \node[title] at (b12.north west) {\underline{Existing Alternatives}};

    % 4. SOLUTION
    \node[bloc2] (b2) at (2*\layerwidth/10,5*\layersep) {
      Outline a possible solution for each problem.
    };
    \node[title] at (b2.north west) {\underline{4. Solution}};

    % 8. KEY METRICS
    \node[bloc2] (b3) at (2*\layerwidth/10,3*\layersep) {
      List the key numbers that tell you how your business is doing.
    };
    \node[title] at (b3.north west) {\underline{8. Key Metrics}};

    % 7. COST STRUCTURE
    \node[bloc3] (b4) at (1.5*\layerwidth/10,1.5*\layersep) {
      List your fixed and variable costs.
    };
    \node[title] at (b4.north west) {\underline{7. Cost Structure}};

    % 6. REVENUE STREAMS
    \node[bloc3] (b5) at (6.5*\layerwidth/10,1.5*\layersep) {
      List your sources of revenue.
    };
    \node[title] at (b5.north west) {\underline{6. Revenue Streams}};

    % 3. VALUE PROPOSITIONS
    \node[bloc1] (b6) at (4*\layerwidth/10,4*\layersep) {};
    \node[title] at (b6.north west) {\underline{3. Value Propositions}};
    
    \node[invisible_bloc2] (b61) at (4*\layerwidth/10,4.45*\layersep) {
      Single, clear, compelling message that states why you are different and worth paying attention.
    };
    
    \node[invisible_bloc1] (b62) at (4*\layerwidth/10,2.5*\layersep) {
      List your X for Y analogy, e.g. YouTube = Flickr for videos.
    };
    \node[title] at (b62.north west) {\underline{High-Level Concept}};

    % 9. UNFAIR ADVANTAGE
    \node[bloc2] (b7) at (6*\layerwidth/10,5*\layersep) {
      Something that cannot easily be bought or copied.
    };
    \node[title] at (b7.north west) {\underline{9. Unfair Advantage}};

    % 5. CHANNELS
    \node[bloc2] (b8) at (6*\layerwidth/10,3*\layersep) {
      List your path to customers (inbound or outbound).
    };
    \node[title] at (b8.north west) {\underline{5. Channels}};


    % 2. CUSTOMER SEGMENTS
    \node[bloc1] (b9) at (8*\layerwidth/10,4*\layersep) {};
    \node[title] at (b9.north west) {\underline{2. Customer Segments}};
    
    \node[invisible_bloc2] (b91) at (8*\layerwidth/10,4.45*\layersep) {
      List your target customers and users.
    };
    
    \node[invisible_bloc1] (b92) at (8*\layerwidth/10,2.5*\layersep) {
      List the characteristics of your ideal customers.
    };
    \node[title] at (b92.north west) {\underline{Early Adopters}};

  \end{tikzpicture}
}


The Lean Canvas latex code is based on \href{https://github.com/OchoaLab/grantModelCanvas}{Alejandro Ochoa's Grant Model Canvas}. Lean Canvas is based on \href{https://leanstack.com/leancanvas}{www.leancanvas.com}.

\end{document}
